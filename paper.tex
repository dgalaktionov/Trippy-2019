% This is samplepaper.tex, a sample chapter demonstrating the
% LLNCS macro package for Springer Computer Science proceedings;
% Version 2.20 of 2017/10/04
%
\documentclass[runningheads]{llncs}
%
\usepackage{graphicx}
% Used for displaying a sample figure. If possible, figure files should
% be included in EPS format.
%
% If you use the hyperref package, please uncomment the following line
% to display URLs in blue roman font according to Springer's eBook style:
% \renewcommand\UrlFont{\color{blue}\rmfamily}

\usepackage{enumerate}
\usepackage[linesnumbered,commentsnumbered,boxed]{algorithm2e}
\usepackage{marginnote}

\newcommand{\acumm}{T-Matrices} % T is for Tirso!
\newcommand{\ctr}{NewCTR (name subject to change)}

\begin{document}
%
\title{Next Trippy Structures\thanks{Supported by organization x.}}
%
%\titlerunning{Abbreviated paper title}
% If the paper title is too long for the running head, you can set
% an abbreviated paper title here
%
\author{First Author\inst{1}\orcidID{0000-1111-2222-3333} \and
Second Author\inst{2,3}\orcidID{1111-2222-3333-4444} \and
Third Author\inst{3}\orcidID{2222--3333-4444-5555}}
%
\authorrunning{F. Author et al.}
% First names are abbreviated in the running head.
% If there are more than two authors, 'et al.' is used.
%
\institute{Princeton University, Princeton NJ 08544, USA \and
Springer Heidelberg, Tiergartenstr. 17, 69121 Heidelberg, Germany
\email{lncs@springer.com}\\
\url{http://www.springer.com/gp/computer-science/lncs} \and
ABC Institute, Rupert-Karls-University Heidelberg, Heidelberg, Germany\\
\email{\{abc,lncs\}@uni-heidelberg.de}}
%
\maketitle              % typeset the header of the contribution
%
\begin{abstract}
The abstract should briefly summarize the contents of the paper in
15--250 words.

\keywords{First keyword  \and Second keyword \and Another keyword.}
\end{abstract}
%
%
%
\section{Introduction}
We took the idea from our SPIRE paper and we made it great, using ZSTD compression and other tricks to improve the time complexity of the queries for the TTCTR. Regarding AccumM, it was not possible to improve its speed (as it was already a few memory accesses per query), but we managed to compress it without increasing time complexity. We rock.

\section{State of the Art}
Review whatever we may have missed from state of the art in prev. papers. Also explain ZSTD and applications. Finally, explain our work-in-progress model for trip representation.

\subsection{A model for compact trip representation}
So here goes a picture where we have stops $s_i \in S = \{1,...|S|\}$, lines $l_i \in L = \{1,...|L|\}$ and journeys $j_i \in J^l = \{1,...|J^l|\}$. It is important to state that journeys are \textbf{not} identified by $j_i$, as the same $j_i$ can belong to several $J^l$ from different lines, so we speak about journey \textbf{codes} (jcodes) instead of journey identifiers.

Users can be attached to trips, but we don't care about them yet.

\section{Proposed Structures}
\subsection{Common Structures}
We represent the network like this:

\begin{itemize}
    \item $lineStop_i(j)$ is the $j$th stop of line $l_i$
    \item $stopLine_i(j)$ is the $j$th line that makes a stop at the stop $s_i$
    \item $avgTime_i(j)$ is the average time in seconds that it takes for a vehicle of line $l_i$ to reach its $j$th stop from the start of a journey
    \item $initialTime_i(k)$ is the starting time of the journey $j_k$ for line $l_i$
\end{itemize}

All of them stored explicitly except for $initialTime$, which is compressed with sampled ZSTD.

\subsection{\acumm}
Tirso please explain what you did. Did you use bitmaps for partial sums? Matrices? Any other trick from inverted lists? Also explain how queries are solved again, in detail.

\subsection{\ctr}
A user trip can be represented by the stops where the user gets on the transportation system, so from now on we will consider a trip as a sequence of triplets $<s,l,j>$, where $s$, $l$ are, respectively, stop and line identifiers, while $j$ are the journey codes corresponding to the journeys that compose the trip. Additionally, as we are interested in knowing where the trips end, we also represent the last stop where the user got off, which line and journey will logically match the line and journey of the last get-on stop. Although it is generally hard to obtain information about the last destination stop of a trip, many transportation companies are investing effort in providing it, either by implementing systems to keep track of users as they leave their system or estimating it based on previous trips made by that user\marginnote{[CITATION NEEDED]}.

We use three complementary structures to represent each component of the sequence:
\begin{enumerate}[(i)]
    \item CSA with special and circular $\$$
    \item lines WM "forest"
    \item jcodes WM
\end{enumerate}

This construction allows us to answer from x to y with just (i).

Then we can use (ii) to restrict lines.

And finally we can restrict times by finding out the right jcodes and filtering through (iii).

Theoretically we can answer anything. It's a representation, after all...

\begin{algorithm}[H]
\SetKwData{la}{l$_a$}\SetKwData{lz}{l$_z$}\SetKwData{sa}{s$_a$}\SetKwData{sz}{s$_z$}\SetKwData{ta}{t$_a$}\SetKwData{tz}{t$_z$}\SetKwData{pattern}{pattern}\SetKwData{left}{left}\SetKwData{right}{right}\SetKwData{csa}{CSA}\SetKwData{wmj}{WMJ}\SetKwData{leftzero}{left$_0$}\SetKwData{rightzero}{right$_0$}\SetKwData{a}{a}\SetKwData{z}{z}\SetKwData{ja}{j$_a$}\SetKwData{jz}{j$_z$}\SetKwData{n}{n}\SetKwData{ap}{a'}\SetKwData{zp}{z'}\SetKwData{offset}{offset}
 \SetKwFunction{GetRange}{GetRange}\SetKwFunction{GetJCodes}{GetJCodes}\SetKwFunction{GetCount}{GetCount}\SetKwFunction{GetPsi}{$\Psi$}\SetKwFunction{GetRangeSpecial}{GetRange$^*$}\SetKwFunction{wml}{WML}\SetKwFunction{TrackUp}{TrackUp}\SetKwFunction{Select}{Select}
 \KwData{\la,\lz,\sa,\sz, times \ta,\tz and length of the sequence \n}
 \KwResult{Number of occurences}
 \BlankLine
 \pattern $\leftarrow \{\sz,0,\sa\}$\;
 \left,\right $\leftarrow$ \GetRange{\csa, $0$, \n, \pattern}\;
 \leftzero $\leftarrow$ \GetPsi{\csa, \left}\;
 \rightzero $\leftarrow$ \GetPsi{\csa, \right}\;
 \tcp{\right-\left = \rightzero-\leftzero}
 \a,\z $\leftarrow$ \GetRange{\wml{0},\leftzero,\rightzero,\la}\;
 \ja,\jz $\leftarrow$ \GetJCodes{\la,\sa,\ta,\tz}\;
 \a,\z $\leftarrow$ \GetRangeSpecial{\wmj,\a,\z,\ja,\jz}\;
 \ap $\leftarrow$ \TrackUp{\wml{0},\a}\;
 \zp $\leftarrow$ \TrackUp{\wml{0},\z}\;
 \tcp{\z-\a = \zp-\ap}
 \offset $\leftarrow$ \Select{\csa,\sz}\;
 \a,\z $\leftarrow$ \GetRange{\wml{\sz},\left-\offset~$+$~\ap-\leftzero,
 \left-\offset~$+$~\ap-\leftzero,\lz}\;
 \ja,\jz $\leftarrow$ \GetJCodes{\lz,\sz,\ta,\tz}\;
 \Return{\GetCount{\wmj,\offset~+~\a,\offset~+~\z,\ja,\jz}}\;
 
 \caption{Querying for all features on \ctr}
\end{algorithm}


\section{Experiments}
For space, I propose to compare with our previous structures and with a plain representation using bits per symbol (bps). Otherwise, the baseline we select for our $\%$ can be misleading.

For time, we will compare against SPIRE 2018, and maybe, IF WE HAVE TIME, with a plain postgresql baseline. Experiments will be executed in Compostela2 or whatever we'll have available.

\section{Conclusions}
Speak about how cool would it be to have a single structure that could to it all. Pitch for our next paper with Gonzalo!

\subsection{A Subsection Sample}
Please note that the first paragraph of a section or subsection is
not indented. The first paragraph that follows a table, figure,
equation etc. does not need an indent, either.

Subsequent paragraphs, however, are indented.

\subsubsection{Sample Heading (Third Level)} Only two levels of
headings should be numbered. Lower level headings remain unnumbered;
they are formatted as run-in headings.

\paragraph{Sample Heading (Fourth Level)}
The contribution should contain no more than four levels of
headings. Table~\ref{tab1} gives a summary of all heading levels.

\begin{table}
\caption{Table captions should be placed above the
tables.}\label{tab1}
\begin{tabular}{|l|l|l|}
\hline
Heading level &  Example & Font size and style\\
\hline
Title (centered) &  {\Large\bfseries Lecture Notes} & 14 point, bold\\
1st-level heading &  {\large\bfseries 1 Introduction} & 12 point, bold\\
2nd-level heading & {\bfseries 2.1 Printing Area} & 10 point, bold\\
3rd-level heading & {\bfseries Run-in Heading in Bold.} Text follows & 10 point, bold\\
4th-level heading & {\itshape Lowest Level Heading.} Text follows & 10 point, italic\\
\hline
\end{tabular}
\end{table}


\noindent Displayed equations are centered and set on a separate
line.
\begin{equation}
x + y = z
\end{equation}
Please try to avoid rasterized images for line-art diagrams and
schemas. Whenever possible, use vector graphics instead (see
Fig.~\ref{fig1}).

\begin{figure}
\includegraphics[width=\textwidth]{fig1.eps}
\caption{A figure caption is always placed below the illustration.
Please note that short captions are centered, while long ones are
justified by the macro package automatically.} \label{fig1}
\end{figure}

\begin{theorem}
This is a sample theorem. The run-in heading is set in bold, while
the following text appears in italics. Definitions, lemmas,
propositions, and corollaries are styled the same way.
\end{theorem}
%
% the environments 'definition', 'lemma', 'proposition', 'corollary',
% 'remark', and 'example' are defined in the LLNCS documentclass as well.
%
\begin{proof}
Proofs, examples, and remarks have the initial word in italics,
while the following text appears in normal font.
\end{proof}
For citations of references, we prefer the use of square brackets
and consecutive numbers. Citations using labels or the author/year
convention are also acceptable. The following bibliography provides
a sample reference list with entries for journal
articles~\cite{ref_article1}, an LNCS chapter~\cite{ref_lncs1}, a
book~\cite{ref_book1}, proceedings without editors~\cite{ref_proc1},
and a homepage~\cite{ref_url1}. Multiple citations are grouped
\cite{ref_article1,ref_lncs1,ref_book1},
\cite{ref_article1,ref_book1,ref_proc1,ref_url1}.
%
% ---- Bibliography ----
%
% BibTeX users should specify bibliography style 'splncs04'.
% References will then be sorted and formatted in the correct style.
%
% \bibliographystyle{splncs04}
% \bibliography{mybibliography}
%
\begin{thebibliography}{8}
\bibitem{ref_article1}
Author, F.: Article title. Journal \textbf{2}(5), 99--110 (2016)

\bibitem{ref_lncs1}
Author, F., Author, S.: Title of a proceedings paper. In: Editor,
F., Editor, S. (eds.) CONFERENCE 2016, LNCS, vol. 9999, pp. 1--13.
Springer, Heidelberg (2016). \doi{10.10007/1234567890}

\bibitem{ref_book1}
Author, F., Author, S., Author, T.: Book title. 2nd edn. Publisher,
Location (1999)

\bibitem{ref_proc1}
Author, A.-B.: Contribution title. In: 9th International Proceedings
on Proceedings, pp. 1--2. Publisher, Location (2010)

\bibitem{ref_url1}
LNCS Homepage, \url{http://www.springer.com/lncs}. Last accessed 4
Oct 2017
\end{thebibliography}
\end{document}
